\documentclass{dcbl/challenge}

\setdoctitle{Expressions and Functions}
\setdocauthor{Stephan Bökelmann}
\setdocemail{sboekelmann@ep1.rub.de}
\setdocinstitute{AG Physik der Hadronen und Kerne}
\usepackage{amsfonts}
\usepackage{amsmath}
\usepackage{listings}


\begin{document}

The basic purpose of our computing-machines is to perform calculations.
But not in the sense of abstract mathematics, like rearranging polynomials, or solving equations, but in the sense of evaluating expressions.
An expression is a mathematical object that can be evaluated to produce a value. 
It consists of operands and operators that dictate how these operands are processed or combined. 
Expressions are fundamental to programming and mathematical logic, allowing for the representation of calculations, conditions, and data transformations within a program. They can range from simple arithmetic calculations to complex function calls and conditional evaluations. 
When an expression is evaluated, it produces a single value, which can then be used in further operations, assigned to variables, or used to make decisions in the flow of a program.


\section*{Exercises}
\begin{aufgabe}
    By combining expressions, we can create more complex functions.
    These functions can be encoded into our C-code, by using appropriate syntax.
    Let's take the following function:\\
    \(f: \{z \in \mathbb{N}_0 \mid 0 \leq z \leq \text{{\texttt{UINT\_MAX}}}\} \rightarrow \mathbb{N}_0\) defined by \(f(x) = \left\lfloor \frac{x}{2} \right\rfloor + \left\lfloor \frac{x}{4} \right\rfloor\), with \(\text{{\texttt{UINT\_MAX}}}\) the largest element of \text{{\texttt{unsigned int}}} determined by the compiler.
    A function given like that can be implemented in the following way:
    \begin{lstlisting}
    unsigned int f(unsigned int x) {
        return (x / 2) + (x / 4);
    }
    \end{lstlisting}
    And be called in the following way:
    \begin{lstlisting}
    unsigned int result = f(10);
    \end{lstlisting}
    Explain, how the datatypes of the parameters and return value relate to the domain and range of the function.    
\end{aufgabe}

\begin{aufgabe}
    Implement the following functions in C:
    \begin{enumerate}
        \item \(g: \{ (x, y) \in \mathbb{N}_0 \times \mathbb{N}_0 \mid 0 \leq x, y \leq \text{{\texttt{UINT\_MAX}}}\} \rightarrow \mathbb{N}_0\); \(g(x, y) = \left\lfloor \frac{x + y}{3} \right\rfloor + \left\lfloor \frac{x + y}{5} \right\rfloor\), with \(\text{{\texttt{UINT\_MAX}}}\) the largest element of \text{{\texttt{unsigned int}}} determined by the compiler.
        \item \(h: \mathbb{Z}^3 \rightarrow \mathbb{R}\); \[h(x, y, z) = \sqrt{x^2 + y^2 + z^2}\] with \(x, y, z \in \mathbb{Z}\) and the result in \( \mathbb{R} \) approximated by a floating point number.
    \end{enumerate}
\end{aufgabe}

%\section*{Annotations}
%\begin{enumerate}
%    \item Link to a youtube video: \url{https://www.youtube.com}
%\end{enumerate}

\end{document}
